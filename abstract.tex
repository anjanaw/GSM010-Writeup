\section*{Abstract}

Musculoskeletal Disorders (MSD) are chronic conditions which have a long term impact on individuals as well as on the community. They require self-management, typically in the form of maintaining an active lifestyle that adherence to prescribed exercises regimes. Therefore it is important to raise awareness about the importance of adherence and accordingly provide necessary support mechanisms to discourage sedentary lifestyles. In the recent past m-health applications gained popularity by gamification of physical activity monitoring and has had a positive impact on general health and well-being. However maintaining a regular exercise routine with correct execution needs more sophistication in human movement recognition compared to monitoring ambulatory activities (such as walking and running). In this research we propose a digital intervention which can intercept, recognize and evaluate exercises in real-time with a view to supporting exercise self-management plans. We will employ multiple sensors of different modalities (accelerometer, pressure and RGB+D sensors) to intercept exercises then implement reasoning methods to recognize and evaluate performance quality. We plan to compile a multi-sensor spatio-temporal dataset for exercises, then we will improve upon state of the art deep learning models to reason with that data. We expect our research contributions to be in the area of deep learning with focus on multi-sensor fusion algorithms and similarity comparison methodologies for spatio-temporal data. Importantly this work involves collaboration with the health sciences researchers to collect multi-modal exercise datasets which will be made public for comparative studies in human activity and exercise recognition research.