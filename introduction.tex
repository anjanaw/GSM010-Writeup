\section{Introduction}

Years lived with disability (YLD) is a measure of overall disease burden in a country and \citeasnoun{abajobir2017global} suggest one of the primary causes for YLD in developed countries is recognized as musculoskeletal disorders (MSDs). Maintaining a regular self-managed exercise routine is an essential component when living with MSDs. Specifically for elderly people (e.g. arthritis and hip-replacement) and many other chronic diseases (e.g. low-back pain, diabetes) it is important to maintain such routines and importantly to adhere to correct execution. Finding technological solutions to support either the prevention or self-management of MSDs has been a research area which has emerged over the last few years.

Trust in E-health digital interventions has grown in the healthcare community as increasingly effective methods are adopted with significant benefit realised in many disciplines (e.g. stroke prevention \cite{widmer2015digital}, self-management of asthma \cite{mclean2016interactive}, depression \cite{li2014game}). Digital interventions have also proven to encourage active lifestyle with the rise of smart devices \cite{fanning2012increasing}. For instance activity monitoring applications using smart devices collect data that help to keep track of user activities based on sensors that can be worn as well as those that are in the environment. For instance user activities such walking can now be captured through automated step counting with accelerometer sensors commonly found in mobile phones or watches. In contrast keeping track of exercises is completely relies on user input, resulting in poor accuracy and reliability. Therefore a digital intervention which captures exercises as they are performed and provides real-time feedback on performance quality would help to motivate the user to adhere to a regular exercise routine while supporting the collection of monitoring data crucial for the delivery of effective interventions.

An effective Digital intervention for self managing MSDs should consist of three main components: intercepting exercises in real-time to enable the automated data collection;  recognition of the exercise being performed; and the evaluation of the quality of the performance to facilitate personalised feedback generation. Today there are sophisticated sensors with improved sensing accuracy and efficient power consumption and they are the most recognized medium to intercepting human activities. Simple sensors on a smart phone are able to identify simple ambulatory activities. But an exercise is a sequence of independent movements of multiple body parts hence a smart watch on the wrist will not be able to capture an exercise with the level of precision required; furthermore a sensor is also likely to be susceptible to noise (due to a greater degree of freedom) not related to the exercise. Therefore it is evident that exercises require capturing different perspectives from multiple sensors, thus pose challenges of reasoning with multiple sensor modalities. 

Exercise recognition can be viewed as a special case of Human Activity Recognition (HAR). Research in this area involves the use of machine learning methods and more recently deep learning algorithms to reason with sensor data. \citeasnoun{yao2017deepsense} and \citeasnoun{hammerla2016deep} show that Convolutional Neural Networks, Recurrent Neural Networks and Deep hybrid architectures have been explored in this regard and yielded significantly better results when compared with traditional classification methods or Deep Neural Networks. We also explore the use of these architectures and our initial experiments (in Appendix \ref{appendix:selfback}) suggest that combining sensor modalities does not necessarily improve reasoning accuracy. Therefore it is important to explore sensor combinations and configurations to support improved reasoning with multi-modal data.  

In the rest of this document we will present literature in Section \ref{sec:literature} and then conclude with gaps that drive our research. Thereafter we propose research questions in Section \ref{sec:rq} and list expected outcomes in Section \ref{sec:outcomes} followed by proposed methods in Section \ref{sec:methods} to address our research questions. 