\section{Overview of ethical issues}
A multi-modal spatio-temporal data collection for exercise classification and quality assessment is one of the main outcomes of this project. Dataset will comprise of sensor data recorded for list of exercises performed by number of participants. Compiling such dataset involves ethical issues in data collection and distribution.

First there are ethical considerations regarding the participants of the data collection task. Following guidelines were set for participants to evaluate their eligibility and get their consent.
\begin{itemize}
\item Participants are selected from volunteers aged above 18 years. 
\item Each participant will complete Physical Activity Readiness Questionnaire (PAR-Q) which will evaluate their fitness to perform the list of exercises. 
\end{itemize}
Next there are medical concerns that may occur during data collection. We set following guidelines to mitigate such situations.
\begin{itemize}
\item Hypo-allergenic tape will be used to minimize the reaction to adhesive tape used to fix sensors. If reaction occurs the participant will be stopped and sensors will be removed, if reaction fail to settle participant will be advised to seek medical attention. 
\item Participant will be stopped if any discomfort occur during data collection, also will be advised to consult a health care practitioner if discomfort fail to settle. 
\end{itemize}
Lastly there are ethical consideration regarding the data storage and distribution. 
\begin{itemize}
\item During data collection, data will be stored in a secure RGU server with restricted access.
\item At the completion of data collection, data will be anonymized to remove all participant identification information.
\item Anonymized dataset will be made public with publications based on the dataset. 
\end{itemize}
Above measures were presented and approved by the School Research Review Group of School of Health Sciences (Appendix \ref{appendix:ethics}).