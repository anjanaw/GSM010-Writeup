\section{Research Questions and Objectives}
\label{sec:research_questions}
This research will explore reasoning with multiple sensors. The overall goal is to introduce novel components for a sustainable digital intervention for self-managing exercise routines. Following review of literature we identity research questions to achieve aforementioned goal . 

\paragraph*{Research Questions}
\label{sec:rq}
\begin{enumerate}
\item [RQ1] How to combine multi-modal data streams to improve exercise recognition?
\item [RQ2] How to maintain recognition accuracy in the presence of noisy and/or missing sensor modalities?
\item [RQ3] How to analyse exercise performance quality by comparing actual and expected multi-modal sensor data?
\end{enumerate}

\noindent We hope to meet following objectives in order to answer each research question. 

\paragraph*{Objectives}
\label{sec:outcomes}

\begin{itemize}
\item [O1] Compile a multi-modal sensor dataset in the domain of exercises with HAI using three sensing modalities. 

\item [O2] Develop a sensor fusion architecture with a dynamic attention layer to manage the combination of multi-modal sensor data to improve recognition accuracy. 

\item [O3] Implement methodologies to handle absence of modalities in real-time. 
This would enable the network to learn with all modalities but remain robust even with fewer modalities in real time. 

\item [O4] Introduce a similarity based architecture for comparing sensor data instances to generate a quality assessment. The resulting solution should localize performance problem to lower level actions of the exercise. 
\end{itemize}