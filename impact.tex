\section{Identified Impact}

This research will introduce new state of the art methodologies for reasoning with multi-modal sensor streams. Machine learning research has evolved rapidly with the introduction of deep learning techniques build upon the support of tremendous computational power. We plan to build upon current deep learning techniques on HAR, adapt and improve them for multi-modal sensor reasoning for exercises. With objective 1 we will bring together different machine learning data domains such as time series, images and video as we create a multi-modal spatio-temporal dataset. It will be adaptable to different application domains as well as future research in HAI. It will also contribute towards evaluating transfer learning capabilities of machine learning models. Objective 2 will introduce new attention mechanisms for sensor data selection in abstract levels and in objective 3 we will introduce strategies to improve robustness in deep learning architectures.Objective 4 will implement novel deep learning models for qualitative evaluation which involves similarity comparison of spatio-temporal data. 

From the healthcare application perspective this research will contribute towards a sustainable digital intervention for MSD prevention and self-management while raising awareness. MSD has directly affects the workforce of a country with sedentary lifestyles and as a result it has a major impact on economical and social status of the country. This digital intervention will enable the user to perform physiotherapist recommended exercises at home with supervision from the qualitative evaluation component. We plan to involve users in each step of this research and raise awareness. We will seek user feedback on the utility of the digital intervention to support, maintain and encourage an active lifestyle in the prevention and management of MSDs.

